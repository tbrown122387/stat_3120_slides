\documentclass{beamer}

\mode<presentation> {

%\usetheme{default}
%\usetheme{AnnArbor}
%\usetheme{Antibes}
%\usetheme{Bergen}
%\usetheme{Berkeley}
%\usetheme{Berlin}
%\usetheme{Boadilla}
%\usetheme{CambridgeUS}
%\usetheme{Copenhagen}
%\usetheme{Darmstadt}
%\usetheme{Dresden}
%\usetheme{Frankfurt}
%\usetheme{Goettingen}
%\usetheme{Hannover}
%\usetheme{Ilmenau}
%\usetheme{JuanLesPins}
%\usetheme{Luebeck}
\usetheme{Madrid}
%\usetheme{Malmoe}
%\usetheme{Marburg}
%\usetheme{Montpellier}
%\usetheme{PaloAlto}
%\usetheme{Pittsburgh}
%\usetheme{Rochester}
%\usetheme{Singapore}
%\usetheme{Szeged}
%\usetheme{Warsaw}


%\usecolortheme{albatross}
%\usecolortheme{beaver}
%\usecolortheme{beetle}
%\usecolortheme{crane}
%\usecolortheme{dolphin}
%\usecolortheme{dove}
%\usecolortheme{fly}
%\usecolortheme{lily}
%\usecolortheme{orchid}
%\usecolortheme{rose}
%\usecolortheme{seagull}
%\usecolortheme{seahorse}
%\usecolortheme{whale}
%\usecolortheme{wolverine}

%\setbeamertemplate{footline} % To remove the footer line in all slides uncomment this line
%\setbeamertemplate{footline}[page number] % To replace the footer line in all slides with a simple slide count uncomment this line

%\setbeamertemplate{navigation symbols}{} % To remove the navigation symbols from the bottom of all slides uncomment this line
}

\usepackage{graphicx} % Allows including images
\usepackage{booktabs} % Allows the use of \toprule, \midrule and \bottomrule in tables
\usepackage{amsfonts}
\usepackage{mathrsfs}
\usepackage{amsmath,amssymb,graphicx}
\usepackage{listings}

%----------------------------------------------------------------------------------------
%	TITLE PAGE
%----------------------------------------------------------------------------------------

\title["12.3"]{12.3: Inferences About the Regression Coefficient $\beta_1$}

\author{Taylor} 
\institute[UVA] 
{
University of Virginia \\
\medskip
\textit{} 
}
\date{} 

\begin{document}
%----------------------------------------------------------------------------------------

\begin{frame}
\titlepage 
\end{frame}
%----------------------------------------------------------------------------------------
\begin{frame}
\frametitle{Introduction}

When we estimate the $\beta$s, we usually are much more interested in $\beta_1$. If this number is non-zero, then there is a relationship between the inputs and the outputs. 
\newline

Our estimate $\hat{\beta}_1$ is based on random data, so it is random itself. $\hat{\beta}_1$ is a point estimator for $\beta_1$. We can do a lot of the same stuff we did when we were estimating $\mu$ with $\bar{X}$.

\end{frame}
%----------------------------------------------------------------------------------------
\begin{frame}
\frametitle{Notation}

Assuming all the $X$s are known/nonrandom, then we can write the estimated slope coefficient like this
\[
\hat{\beta}_1 = \frac{\sum (x_i - \bar{x})(Y_i - \bar{Y})}{\sum(x_i - \bar{x})^2}
\]

The book is capitalizing the $Y$s to emphasize that this is the source if randomness.

\end{frame}
%----------------------------------------------------------------------------------------
\begin{frame}
\frametitle{A Little Trick}

Call $\sum(x_i - \bar{x})^2 = S_{xx}$
\begin{align*}
\hat{\beta}_1 &= \frac{\sum (x_i - \bar{x})(Y_i - \bar{Y})}{S_{xx}} \\
&= \frac{\sum (x_i - \bar{x})Y_i  - \sum (x_i - \bar{x})\bar{Y}}{S_{xx}}\\
&= \frac{\sum (x_i - \bar{x})Y_i  }{S_{xx}} \\
&= \sum_i c_i Y_i
\end{align*}

because $\sum(x_i - \bar{x}) = 0$. This is a linear combination of independent (but not identical) normal rvs.

\end{frame}
%----------------------------------------------------------------------------------------
\begin{frame}
\frametitle{A Distribution for $\hat{\beta}_1$}

So 
\begin{enumerate}
\item $\hat{\beta}_1 \sim \mathcal{N}(\beta_1, \frac{\sigma^2}{S_{xx}})$
\item $\frac{(n-2) \hat{\sigma}^2}{\sigma^2} \sim \chi^2_{n-2}$
\item $\hat{\beta}_1$ is independent from $\frac{(n-2) \hat{\sigma}^2}{\sigma^2}$
\end{enumerate}

where $\hat{\sigma}^2 = \sum_i\frac{(y_i - \hat{y}_i)^2}{n-2}$
\newline


so that means...
\end{frame}
%----------------------------------------------------------------------------------------
\begin{frame}
\frametitle{Our distribution}

\[
T = \left[ \frac{\hat{\beta}_1 - \beta_1}{\frac{\sigma}{\sqrt{S_{xx}} }}\right] \div \left[\sqrt{\frac{(n-2) \hat{\sigma}^2}{\sigma^2 (n-2)} }\right] = \frac{\hat{\beta}_1 - \beta_1}{\hat{\sigma} / \sqrt{S_{xx}}}
\]

follows a $t_{n-2}$ distribution.
\end{frame}
%----------------------------------------------------------------------------------------
\begin{frame}
\frametitle{A CI for $\beta_1$}

Based on
\[
P\left( - t_{\alpha/2,n-2} \le \frac{\hat{\beta}_1 - \beta_1}{\hat{\sigma} / \sqrt{S_{xx}}} \le t_{\alpha/2,n-2} \right) = 1-\alpha
\]
we can do a bit of arithmetic and 
\[
\hat{\beta}_1 \pm t_{\alpha/2,n-2} \frac{\hat{\sigma} }{ \sqrt{S_{xx}}}
\]
\end{frame}
%----------------------------------------------------------------------------------------
\begin{frame}
\frametitle{A Hypothesis Test for $\beta_1$}

\begin{enumerate}
\item $H_0: \beta_1 = \beta_{1,0}$
\item $t = \frac{\hat{\beta}_1 - \beta_{1,0} }{\hat{\sigma} / \sqrt{S_{xx}}}$
\item if $H_a: \beta_1 > \beta_{1,0}$, reject if $t > t_{\alpha,n-2}$
\item if $H_a: \beta_1 < \beta_{1,0}$, reject if $t < -t_{\alpha,n-2}$
\item if $H_a: \beta_1 \neq \beta_{1,0}$, reject if $t > t_{\alpha/2,n-2}$ or if $t < -t_{\alpha/2,n-2}$
\end{enumerate}




The most common situation is when $H_0:\beta_1 = 0$ versus $H_a: \beta_1 \neq 0$. This is basically testing if $X$ is associated (linearly) with $Y$.

\end{frame}
%----------------------------------------------------------------------------------------
\begin{frame}[fragile]
\frametitle{A Note About Fitting Logistic Regression Models}

$Y_i \sim \text{Bernoulli}[p(x_i)], i=1,\ldots,n$. The likelihood is then
\[
\prod_i f(y_i) = \prod_i p(x_i)^{y_i}[1-p(x_i)]^{1 - y_i}
\]

We still fit this using maximum likelihood estimation, but there's no closed-form solution for the coefficients. Also notice how each $p(x_i)$ is different possibly, so we can't combine/simplify stuff.
\newline

Many software packages fit these. In \verb|R| it would be something like 

\begin{verbatim} 
my_mod <- glm(y ~ x, family = "binomial")
\end{verbatim}


\end{frame}
%----------------------------------------------------------------------------------------


\end{document} 
