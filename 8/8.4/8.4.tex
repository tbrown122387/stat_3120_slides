\documentclass{beamer}

\mode<presentation> {

%\usetheme{default}
%\usetheme{AnnArbor}
%\usetheme{Antibes}
%\usetheme{Bergen}
%\usetheme{Berkeley}
%\usetheme{Berlin}
%\usetheme{Boadilla}
%\usetheme{CambridgeUS}
%\usetheme{Copenhagen}
%\usetheme{Darmstadt}
%\usetheme{Dresden}
%\usetheme{Frankfurt}
%\usetheme{Goettingen}
%\usetheme{Hannover}
%\usetheme{Ilmenau}
%\usetheme{JuanLesPins}
%\usetheme{Luebeck}
\usetheme{Madrid}
%\usetheme{Malmoe}
%\usetheme{Marburg}
%\usetheme{Montpellier}
%\usetheme{PaloAlto}
%\usetheme{Pittsburgh}
%\usetheme{Rochester}
%\usetheme{Singapore}
%\usetheme{Szeged}
%\usetheme{Warsaw}


%\usecolortheme{albatross}
%\usecolortheme{beaver}
%\usecolortheme{beetle}
%\usecolortheme{crane}
%\usecolortheme{dolphin}
%\usecolortheme{dove}
%\usecolortheme{fly}
%\usecolortheme{lily}
%\usecolortheme{orchid}
%\usecolortheme{rose}
%\usecolortheme{seagull}
%\usecolortheme{seahorse}
%\usecolortheme{whale}
%\usecolortheme{wolverine}

%\setbeamertemplate{footline} % To remove the footer line in all slides uncomment this line
%\setbeamertemplate{footline}[page number] % To replace the footer line in all slides with a simple slide count uncomment this line

%\setbeamertemplate{navigation symbols}{} % To remove the navigation symbols from the bottom of all slides uncomment this line
}

\usepackage{graphicx} % Allows including images
\usepackage{booktabs} % Allows the use of \toprule, \midrule and \bottomrule in tables
\usepackage{amsfonts}
\usepackage{mathrsfs}
\usepackage{amsmath,amssymb,graphicx}

%----------------------------------------------------------------------------------------
%	TITLE PAGE
%----------------------------------------------------------------------------------------

\title["8.4"]{8.4: Confidence Intervals for the Variance and Standard Deviation of a Normal Population}

\author{Taylor} 
\institute[UVA] 
{
University of Virginia \\
\medskip
\textit{} 
}
\date{} 

\begin{document}
%----------------------------------------------------------------------------------------

\begin{frame}
\titlepage 
\end{frame}
%----------------------------------------------------------------------------------------

\begin{frame}
\frametitle{Motivation}
 
We have the same set up as the last chapter (that is, a normal random sample). Now instead of being concerned about $\mu$, we're concerned about $\sigma$ or $\sigma^2$.

\end{frame}
%----------------------------------------------------------------------------------------

\begin{frame}
\frametitle{The set-up}
 
Recall that if $X_1, \ldots, X_n \overset{iid}{\sim} \mathcal{N}(\mu, \sigma^2)$, then
\[
\frac{(n-1)S^2}{\sigma^2} \sim \chi^2_{n-1}.
\]

Now let 
\[
\chi^2_{\alpha, \nu}
\]
denote the $(1-\alpha)$100th percentile of the $\chi^2_{\nu}$ distribution.
\newline

It is important to remember that the $\chi^2_{\nu}$ distribution isn't symmetric, so $\chi^2_{1 - \alpha, \nu} \neq -\chi^2_{\alpha, \nu}$. We have to find two quantiles per problem this time.

\end{frame}
%----------------------------------------------------------------------------------------


\begin{frame}
\frametitle{Example}

But the idea is the same as in previous chapters:
\[
P \left( \chi^2_{1 - \alpha/2, n-1} \le \frac{(n-1)S^2}{\sigma^2} \le \chi^2_{\alpha/2, n-1} \right) = 1- \alpha
\]
or
\[
P \left( \sigma^2 \chi^2_{1 - \alpha/2, n-1} \le (n-1)S^2 \le \sigma^2 \chi^2_{\alpha/2, n-1} \right) = 1- \alpha
\]
or
\[
P \left( \sigma^2  \le \frac{(n-1)S^2}{\chi^2_{1 - \alpha/2, n-1}} \text{  and  } \sigma^2 \ge \frac{(n-1)S^2}{\chi^2_{ \alpha/2, n-1}} \right) = 1- \alpha
\]
So our $(1-\alpha)$100th percentile for $\sigma^2$ is $\left[ \frac{(n-1)S^2}{\chi^2_{ \alpha/2, n-1}}, \frac{(n-1)S^2}{\chi^2_{1 - \alpha/2, n-1}} \right]$
\end{frame}
%----------------------------------------------------------------------------------------


\end{document} 
