\documentclass{beamer}

\mode<presentation> {

%\usetheme{default}
%\usetheme{AnnArbor}
%\usetheme{Antibes}
%\usetheme{Bergen}
%\usetheme{Berkeley}
%\usetheme{Berlin}
%\usetheme{Boadilla}
%\usetheme{CambridgeUS}
%\usetheme{Copenhagen}
%\usetheme{Darmstadt}
%\usetheme{Dresden}
%\usetheme{Frankfurt}
%\usetheme{Goettingen}
%\usetheme{Hannover}
%\usetheme{Ilmenau}
%\usetheme{JuanLesPins}
%\usetheme{Luebeck}
\usetheme{Madrid}
%\usetheme{Malmoe}
%\usetheme{Marburg}
%\usetheme{Montpellier}
%\usetheme{PaloAlto}
%\usetheme{Pittsburgh}
%\usetheme{Rochester}
%\usetheme{Singapore}
%\usetheme{Szeged}
%\usetheme{Warsaw}


%\usecolortheme{albatross}
%\usecolortheme{beaver}
%\usecolortheme{beetle}
%\usecolortheme{crane}
%\usecolortheme{dolphin}
%\usecolortheme{dove}
%\usecolortheme{fly}
%\usecolortheme{lily}
%\usecolortheme{orchid}
%\usecolortheme{rose}
%\usecolortheme{seagull}
%\usecolortheme{seahorse}
%\usecolortheme{whale}
%\usecolortheme{wolverine}

%\setbeamertemplate{footline} % To remove the footer line in all slides uncomment this line
%\setbeamertemplate{footline}[page number] % To replace the footer line in all slides with a simple slide count uncomment this line

%\setbeamertemplate{navigation symbols}{} % To remove the navigation symbols from the bottom of all slides uncomment this line
}

\usepackage{graphicx} % Allows including images
\usepackage{booktabs} % Allows the use of \toprule, \midrule and \bottomrule in tables
\usepackage{amsfonts}
\usepackage{mathrsfs}
\usepackage{amsmath,amssymb,graphicx}

%----------------------------------------------------------------------------------------
%	TITLE PAGE
%----------------------------------------------------------------------------------------

\title["10.1"]{10.1: Z Tests and Confidence Intervals for a Difference Between Two Population Means}

\author{Taylor} 
\institute[UVA] 
{
University of Virginia \\
\medskip
\textit{} 
}
\date{} 

\begin{document}
%----------------------------------------------------------------------------------------

\begin{frame}
\titlepage 
\end{frame}
%----------------------------------------------------------------------------------------
\begin{frame}
\frametitle{Introduction}

Now we'll have two datasets $X_1, \ldots, X_m$ and $Y_1, \ldots, Y_n$. We'll do confidence intervals and hypothesis testing.
\newline

Recall we had three cases when we were dealing with hypothesis testing/confidence intervals for single samples. Here we will only have two. We don't get that same third case with two samples. This is because there are some complications when dealing with $t$ rvs. That's why next chapter is devoted to that. So, just two cases here.

\end{frame}
%----------------------------------------------------------------------------------------
\begin{frame}
\frametitle{Introduction}

Most often we want to test if $\mu_1 = \mu_2$. Or equivalently $\mu_1 - \mu_2 = 0$. We're going to use the second one more often because we can think about this difference in parameters as a single thing we can make hypotheses about. We can also test if the difference is some value other than zero.
\newline

Basic assumptions:
\begin{enumerate}
\item $X_1, \ldots, X_m$ is a random sample from a population with mean $\mu_1$ and variance $\sigma^2_1$.
\item $Y_1, \ldots, Y_n$ is a random sample from a population with mean $\mu_2$ and variance $\sigma^2_2$.
\item The $X$ and $Y$ samples are independent from each other.
\end{enumerate}

\end{frame}
%----------------------------------------------------------------------------------------
\begin{frame}
\frametitle{Introduction}

$\bar{X} - \bar{Y}$ is a natural estimator for $\mu_1 - \mu_2$.
\newline


\begin{enumerate}
\item $\bar{X} - \bar{Y}$ is unbiased
\item $SE_{actual}(\bar{X} - \bar{Y}) = \sqrt{\frac{\sigma_1^2}{m}+\frac{\sigma_2^2}{n} } $
\item $SE_{estimated}(\bar{X} - \bar{Y}) = \sqrt{\frac{s_1^2}{m}+\frac{s_2^2}{n} } $
\end{enumerate}


\end{frame}
%----------------------------------------------------------------------------------------
\begin{frame}
\frametitle{Introduction}

So 
\[
Z = \frac{\bar{X} - \bar{Y} - (\mu_1 - \mu_2)}{\sqrt{\frac{\sigma_1^2}{m}+\frac{\sigma_2^2}{n} } } \sim \mathcal{N}(0, 1).
\]
We use this as our test statistic motivation when we know the actual standard error. It follows a normal distribution if our data are normally distributed. Or if that isn't true, it still works because we can use the CLT.




\end{frame}
%----------------------------------------------------------------------------------------
\begin{frame}
\frametitle{Synopsis}

\begin{enumerate}
\item $H_0: \mu_1 - \mu_2 = \Delta_0$
\item $z = \frac{\bar{x} - \bar{y} - \Delta_0}{\sqrt{\frac{\sigma_1^2}{m}+\frac{\sigma_2^2}{n} } }$
\item If $H_a: \mu_1 - \mu_2 > \Delta_0$ reject when $z \ge z_{\alpha}$
\item If $H_a: \mu_1 - \mu_2 < \Delta_0$ reject when $z \le -z_{\alpha}$
\item If $H_a: \mu_1 - \mu_2 \neq \Delta_0$ reject when $z \ge z_{\alpha/2}$ OR when $z \le -z_{\alpha/2}$
\end{enumerate}

\end{frame}
%----------------------------------------------------------------------------------------
\begin{frame}
\frametitle{Correlation is not Causation}

Side Note: has a few pages on how rejection of $H_0: \mu_1 - \mu_2 = 0$ doesn't allow us to say anything about causality. It defines \textbf{observational} studies, \textbf{retrospective} studies, and \textbf{randomized controlled experiments}. We won't talk about this, but it might be worth looking at if you're interested in domains where this is a big issue.

\end{frame}
%----------------------------------------------------------------------------------------
\begin{frame}
\frametitle{Type 2 Error and Sample Size Determination}

Let's say $H_0: \mu_1 - \mu_2 = \Delta_0$ versus $H_a: \mu_1 - \mu_2 > \Delta_0$. Let $\Delta'$ denote the true difference in means. Also we write $SE(\cdot)$ instead of $SE_{actual}(\cdot)$. From now on we'll assume the context is clear.
\newline

\begin{align*}
\beta(\Delta') &= P\left( \frac{\bar{X} - \bar{Y} - \Delta_0}{\text{SE}(\bar{X} - \bar{Y})} < z_{\alpha}  | \Delta = \Delta' \right) \\
&= P\left( \bar{X} - \bar{Y}   < z_{\alpha}\text{SE}(\bar{X} - \bar{Y}) + \Delta_0  | \Delta = \Delta' \right) \\
&= P\left( \frac{\bar{X} - \bar{Y} - \Delta'}{\text{SE}(\bar{X} - \bar{Y})} < z_{\alpha} + \frac{\Delta_0 - \Delta'}{\text{SE}(\bar{X} - \bar{Y})} \right) \\
&= \Phi\left( z_{\alpha} + \frac{\Delta_0 - \Delta'}{\text{SE}(\bar{X} - \bar{Y})} \right)
\end{align*}

power calculations for other types of hypotheses are listed on page 489.
\end{frame}
%----------------------------------------------------------------------------------------
\begin{frame}
\frametitle{Large-Sample Tests}

This is where we don't know $SE_{actual}(\bar{X} - \bar{Y})$ but we do know $SE_{estimated}(\bar{X} - \bar{Y})$.
\newline

Under $H_0$
\[
Z = \frac{\bar{X} - \bar{Y} - \Delta_0}{\text{SE}_{estimated}(\bar{X} - \bar{Y})} \sim \mathcal{N}(0,1)
\]

The decision rules are the same as the ones on slide 6. Also, the power calculations are difficult for the same reason we discussed in the single sample case. 
\end{frame}
%----------------------------------------------------------------------------------------
\begin{frame}
\frametitle{Confidence Intervals}
Our confidence intervals are based on either 
\[
Z = \frac{\bar{X} - \bar{Y} - \Delta}{\text{SE}_{actual}(\bar{X} - \bar{Y})} \sim \mathcal{N}(0,1)
\]

or

\[
Z = \frac{\bar{X} - \bar{Y} - \Delta}{\text{SE}_{estimated}(\bar{X} - \bar{Y})} \sim \mathcal{N}(0,1)
\]


\end{frame}
%----------------------------------------------------------------------------------------
\begin{frame}
\frametitle{Motivation}

so we can get our confidence intervals based on this equation
\[
P \left( - z_{\alpha/2} < \frac{\bar{X} - \bar{Y} - \Delta}{\text{SE}_{actual}(\bar{X} - \bar{Y})} < z_{\alpha/2} \right) = 1-\alpha
\]
or this equation:
\[
P \left( - z_{\alpha/2} < \frac{\bar{X} - \bar{Y} - \Delta}{\text{SE}_{estimated}(\bar{X} - \bar{Y})} < z_{\alpha/2} \right) = 1-\alpha.
\]

I'll leave it to you to show that we can have confidence intervals
\[
\bar{X} - \bar{Y} \pm z_{\alpha/2} \text{SE}_{actual}(\bar{X} - \bar{Y})
\]
or
\[
\bar{X} - \bar{Y} \pm z_{\alpha/2} \text{SE}_{estimated}(\bar{X} - \bar{Y})
\]
\end{frame}



\end{document} 
