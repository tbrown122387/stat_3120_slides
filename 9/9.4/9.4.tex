\documentclass{beamer}

\mode<presentation> {

%\usetheme{default}
%\usetheme{AnnArbor}
%\usetheme{Antibes}
%\usetheme{Bergen}
%\usetheme{Berkeley}
%\usetheme{Berlin}
%\usetheme{Boadilla}
%\usetheme{CambridgeUS}
%\usetheme{Copenhagen}
%\usetheme{Darmstadt}
%\usetheme{Dresden}
%\usetheme{Frankfurt}
%\usetheme{Goettingen}
%\usetheme{Hannover}
%\usetheme{Ilmenau}
%\usetheme{JuanLesPins}
%\usetheme{Luebeck}
\usetheme{Madrid}
%\usetheme{Malmoe}
%\usetheme{Marburg}
%\usetheme{Montpellier}
%\usetheme{PaloAlto}
%\usetheme{Pittsburgh}
%\usetheme{Rochester}
%\usetheme{Singapore}
%\usetheme{Szeged}
%\usetheme{Warsaw}


%\usecolortheme{albatross}
%\usecolortheme{beaver}
%\usecolortheme{beetle}
%\usecolortheme{crane}
%\usecolortheme{dolphin}
%\usecolortheme{dove}
%\usecolortheme{fly}
%\usecolortheme{lily}
%\usecolortheme{orchid}
%\usecolortheme{rose}
%\usecolortheme{seagull}
%\usecolortheme{seahorse}
%\usecolortheme{whale}
%\usecolortheme{wolverine}

%\setbeamertemplate{footline} % To remove the footer line in all slides uncomment this line
%\setbeamertemplate{footline}[page number] % To replace the footer line in all slides with a simple slide count uncomment this line

%\setbeamertemplate{navigation symbols}{} % To remove the navigation symbols from the bottom of all slides uncomment this line
}

\usepackage{graphicx} % Allows including images
\usepackage{booktabs} % Allows the use of \toprule, \midrule and \bottomrule in tables
\usepackage{amsfonts}
\usepackage{mathrsfs}
\usepackage{amsmath,amssymb,graphicx}

%----------------------------------------------------------------------------------------
%	TITLE PAGE
%----------------------------------------------------------------------------------------

\title["9.4"]{9.4: P-Values}

\author{Taylor} 
\institute[UVA] 
{
University of Virginia \\
\medskip
\textit{} 
}
\date{} 

\begin{document}
%----------------------------------------------------------------------------------------

\begin{frame}
\titlepage 
\end{frame}
%----------------------------------------------------------------------------------------

\begin{frame}
\frametitle{Definition}

The \textbf{p-value} is the probability of obtaining a value of another test statistic at least as contradictory to $H_0$ as the value you just calculated from the available sample; this probability is calculated under the assumption that $H_0$ is true.
\newline

This should seem sort of complicated. We are comparing a hypothetical, unobserved test statistic, $Z$, and comparing it to our now nonrandom test statistic we just calculated, $z$.


\end{frame}
%----------------------------------------------------------------------------------------
\begin{frame}
\frametitle{Definition}

Say you had the setup: $X_1, \ldots, X_n \overset{iid}{\sim} \text{Normal}(\mu, \sigma^2)$, with $\sigma^2$ known. And $H_0: \mu = \mu_0$ versus $H_a: \mu > \mu_0$.
\newline

And say you calculated your test statistic $z = \frac{(\bar{x} - \mu_0)}{\sqrt{\sigma^2/n}}$ (now lowercase). Then
\[
\text{p-value} = P(Z>z|\mu=\mu_0)
\]


\end{frame}
%----------------------------------------------------------------------------------------



\begin{frame}
\frametitle{Example 9.14 on page 456}

We're doing a large-sample hypothesis test here for the population mean $\mu$. $H_0: \mu = 2.0$ versus $H_a: \mu > 2$. They tell us that $z = \frac{\bar{x} - 2.0}{s/\sqrt{n}} = 3.04$. If this was a question in one of the previous sections, we would find $z_{\alpha}$ and reject if $z > z_{\alpha}$. 
\newline

But let's do it with a p-value now. The values that are at least as contradictory to $H_0$ as the test statistic is are all the numbers to the right of our test statistic. This is because $H_a: \mu > 2$.
\newline

\[
\text{P-Value} = P(Z > 3.04 \text{ when $\mu = 2$ }) = .0012
\]

\end{frame}
%----------------------------------------------------------------------------------------


\begin{frame}
\frametitle{Decision Rule}

A small p-value (near 0) indicates rare-ness. Either your assumption of $H_0$ is absurd, or you just observed some very rare data. 
\newline

If $\alpha$ is our pre-selected probability of type 1 error, the decision rule tells us to reject $H_0$ if and only if our p-value is less than $\alpha$.

\end{frame}
%----------------------------------------------------------------------------------------


\begin{frame}
\frametitle{Summary}

Old decision rule: compare $z$ with $z_{\alpha}$. Or compare $t$ with $t_{n-1,\alpha}$.
\newline

New decision rule: compare p-value with $\alpha$. 
\newline

The p-value is based on the percentile associated with your test statistic. All of this is just two sides of the same coin. The upshot of p-values is that they're more ``report-able." 
\newline

People looking at a statistical analysis can just trust you on your methodology and glance at the p-value; they don't need to know about statistical distributions.
\newline

Also, if $\alpha = .1$, then a p-value of $.000000001$ seems to say a bit more than $.08$. 


\end{frame}
%----------------------------------------------------------------------------------------


\begin{frame}
\frametitle{Summary}

How to calculate p-values:
\begin{enumerate}
\item For an upper-tailed test: $P(Z \ge z)$ OR $P(T_{n-1} \ge t)$ 
\item For a lower-tailed test: $P(Z \le z)$ OR $P(T_{n-1} \le t)$ 
\item For a two-tailed test: $2 \times P(Z \le -|z|)$ OR $2 \times P(T_{n-1} \le -|t|)$ 
\end{enumerate}


\end{frame}
%----------------------------------------------------------------------------------------




\end{document} 
