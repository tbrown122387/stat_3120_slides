\documentclass{beamer}

\mode<presentation> {

%\usetheme{default}
%\usetheme{AnnArbor}
%\usetheme{Antibes}
%\usetheme{Bergen}
%\usetheme{Berkeley}
%\usetheme{Berlin}
%\usetheme{Boadilla}
%\usetheme{CambridgeUS}
%\usetheme{Copenhagen}
%\usetheme{Darmstadt}
%\usetheme{Dresden}
%\usetheme{Frankfurt}
%\usetheme{Goettingen}
%\usetheme{Hannover}
%\usetheme{Ilmenau}
%\usetheme{JuanLesPins}
%\usetheme{Luebeck}
\usetheme{Madrid}
%\usetheme{Malmoe}
%\usetheme{Marburg}
%\usetheme{Montpellier}
%\usetheme{PaloAlto}
%\usetheme{Pittsburgh}
%\usetheme{Rochester}
%\usetheme{Singapore}
%\usetheme{Szeged}
%\usetheme{Warsaw}


%\usecolortheme{albatross}
%\usecolortheme{beaver}
%\usecolortheme{beetle}
%\usecolortheme{crane}
%\usecolortheme{dolphin}
%\usecolortheme{dove}
%\usecolortheme{fly}
%\usecolortheme{lily}
%\usecolortheme{orchid}
%\usecolortheme{rose}
%\usecolortheme{seagull}
%\usecolortheme{seahorse}
%\usecolortheme{whale}
%\usecolortheme{wolverine}

%\setbeamertemplate{footline} % To remove the footer line in all slides uncomment this line
%\setbeamertemplate{footline}[page number] % To replace the footer line in all slides with a simple slide count uncomment this line

%\setbeamertemplate{navigation symbols}{} % To remove the navigation symbols from the bottom of all slides uncomment this line
}

\usepackage{graphicx} % Allows including images
\usepackage{booktabs} % Allows the use of \toprule, \midrule and \bottomrule in tables
\usepackage{amsfonts}
\usepackage{mathrsfs}
\usepackage{amsmath,amssymb,graphicx}

%----------------------------------------------------------------------------------------
%	TITLE PAGE
%----------------------------------------------------------------------------------------

\title["9.1"]{9.1: Hypotheses and Test Procedures}

\author{Taylor} 
\institute[UVA] 
{
University of Virginia \\
\medskip
\textit{} 
}
\date{} 

\begin{document}
%----------------------------------------------------------------------------------------

\begin{frame}
\titlepage 
\end{frame}
%----------------------------------------------------------------------------------------

\begin{frame}
\frametitle{Motivation}

So far we've been talking about estimating a parameter $\theta$. We talked about different point estimates, and different types of intervals.
\newline

``Frequently, however, the objective of an investigationis not to estimate a parameter but to decide which of two contradictory claims about the parameter is correct."
\newline

This sort of thing is called \emph{hypothesis testing}. And a lot of this section will be definitions.

\end{frame}
%----------------------------------------------------------------------------------------


\begin{frame}
\frametitle{Definition}

A \textbf{statistical hypothesis} is a claim or assertion either about the value of a single parameter, about the values of several parameters, or about the form of an entire probability distribution.

\end{frame}
%----------------------------------------------------------------------------------------


\begin{frame}
\frametitle{Definition}

The \textbf{null hypothesis}, denoted $H_0$, is the claim that is initially assumed to be true (the ``prior" belief claim). The \textbf{alternative hypothesis}, denoted by $H_a$ is the assertion that is contradictory to $H_0$.
\newline

We reject $H_0$ in favor of $H_a$ if our data tells us to. If it doesn't, we continue to believe in $H_0$. So our decisions are either reject $H_0$ or fail to reject $H_0$.

\end{frame}
%----------------------------------------------------------------------------------------


\begin{frame}
\frametitle{Definition}

A \textbf{test of hypothesis} is a method for using sample data to decide whether the null hypothesis should be rejected.
\newline

Generally our null hypothesis will involve an $=$
\begin{enumerate}
\item $H_0: \theta = 4$
\end{enumerate}
and our alternative hypothesis will take one of the three forms:
\begin{enumerate}
\item $H_a: \theta > 4$
\item $H_a: \theta < 4$
\item $H_a: \theta \neq 4$
\end{enumerate}

Sometimes we will be more abstract and instead of using a specific number, like $4$, we'll use $\theta_0$. This value is called the \textbf{null value}.

\end{frame}
%----------------------------------------------------------------------------------------


\begin{frame}
\frametitle{Definitions}

A test procedure is specified by two things:
\begin{enumerate}
\item a \textbf{test statistic}, a function of the sample data on which the decision is to be based
\item a \textbf{rejection region}, a set of all the test statistic values for which $H_0$ will be rejected
\end{enumerate}

We reject the null hypothesis $H_0$ if and only if our test statistic falls inside the rejection region.
\end{frame}
%----------------------------------------------------------------------------------------

\begin{frame}
\frametitle{What could go wrong?}

A \textbf{type 1 error} consists of rejecting $H_0$ when it's true...the probablity of which is usually denoted by $\alpha$.
\newline

A \textbf{type 2 error} consists of not rejecting $H_0$ when it's false. The probability of this is usually denoted by $\beta$.

\end{frame}
%----------------------------------------------------------------------------------------


\begin{frame}
\frametitle{Example 9.1 on page 429}

``An automobile model is known to sustain no visible damage 25\% of the time in 10-mph crash tests. A modified bumper design has been proposed in an effort to increase this percentage. Let $p$ denote the proportion of all 10-mph crashes with this new bumper that result in no visible damage. The hypotheses to be tested are $H_0: p = .25$ versus $H_a: p > .25$. The test will be based on an experiment involving $n=20$ independent crashes with prototypes of the new design. Intuitively, $H_0$ should be rejected if a substantial number of the crashes show no damage."

\end{frame}
%----------------------------------------------------------------------------------------

\begin{frame}
\frametitle{Example}

Let's have our test statistic be $X$, the number of crashes with no visible damage. If $H_0$ is true, $p=.25$, and $X \sim \text{Binomial}(20,.25)$.
\newline

They give us a rejection region of $R_8 = \{8,9,10,\ldots\}$.
\newline

\begin{align*}
\alpha = P(\text{ Type 1 Error }) &= P(\text{ Reject when $H_0$ is true }) \\
&= P( X \ge 8 \text{ when } p = .25) \\
&= 1 - P(X \le 7 | p = .25) \\
&= .102
\end{align*}
\end{frame}
%----------------------------------------------------------------------------------------

\begin{frame}
\frametitle{Example continued}

Recall that type 2 error is when $H_0$ is false. In this case $H_a: p > .25$. But this is not specific enough; we need a fixed value of $p$ to be able to calculate cdfs. 
\newline

Let's pick $.3$.
\begin{align*}
\beta(.3) = P(\text{ type 2 error when $p$ is $.3$}) &= P(X < 8 | p = .3) \\
&= P(X \le 7 | p=.3) \\
&= .772
\end{align*}

This says that when $p$ is only slightly above $.25$ ($.3$ in this case), that we're going to have a hard time rejecting $H_0$. 77 \% of the time we will make a mistake here. 

\end{frame}
%----------------------------------------------------------------------------------------
\begin{frame}
\frametitle{Example 9.2 on page 430}

``The drying time of a type of paint under specified test conditions is known to be normally distributed with mean value 75 min and standard deviation 9 min. Chemists have proposed a new additive designed to decrease average drying time. It is believed that drying times with this additive will remain normally distributed with $\sigma = 9$. Because of the expense associated with the additive, evidence should strongly suggest an improvement in average drying time before such a conclusion is adopted. Let $\mu$ denote the true average drying time when the additive is used. The appropriate hypotheses are $H_0: \mu = 75$ versus $H_a: \mu < 75$. Only if $H_0$ can be rejected will the additive be declared successfull and used."


\end{frame}
%----------------------------------------------------------------------------------------

\begin{frame}
\frametitle{Example 9.2 on page 430}

For this problem we have a random sample of size $n=25$. Again, our data are all normally distributed, independently, with mean $\mu$ and $\sigma = 9$ and our hypotheses are $H_0: \mu = 75$ versus $H_a: \mu < 75$. 
\newline

Consider the rejection region $R = (-\infty, 70.8]$. We know that $\bar{X} \sim \mathcal{N}(\mu, 1.8^2)$

\end{frame}
%----------------------------------------------------------------------------------------


\begin{frame}
\frametitle{Example 9.2 on page 430}

\begin{align*}
\alpha &= P(\bar{X} \le 70.8 \text{ when $H_0$ is true })\\
&= P(\bar{X} \le 70.8 | \mu = 75)\\
&= P \left( \frac{\bar{X} - 75}{1.8} \le \frac{70.8 - 75}{1.8} \right) \\
&= \Phi \left( \frac{70.8 - 75}{1.8} \right) \\
&= .01
\end{align*}


this is good.
\end{frame}
%----------------------------------------------------------------------------------------


\begin{frame}
\frametitle{Motivation}

Let's consider when $\mu = 72$, which is part of the alternative hypothesis
\begin{align*}
\beta &= P(\bar{X} > 70.8 | \mu = 72)\\
&= P \left( \frac{\bar{X} - 72}{1.8} > \frac{70.8 - 72}{1.8} \right) \\
&= 1 - \Phi \left( \frac{70.8 - 72}{1.8} \right) \\
&= .7486
\end{align*}

this is bad.
\end{frame}
%----------------------------------------------------------------------------------------


\begin{frame}
\frametitle{Motivation}

For any given sample size and experiment, there is a tradeoff between $\alpha$ and $\beta$ whenever you move around the rejection region. Too far away, and you'll never reject $H_0$ (high type 2), but too close and you'll always accidentally reject (high type 1).

\end{frame}
%----------------------------------------------------------------------------------------

\begin{frame}
\frametitle{Motivation}

Traditionally, one will pick a maximum level for $\alpha$. Then, out of all the possible test choices that fulfill this criterion, one will select the test with the highest power (or lowest type 2 error). So we control type 1 error, and hope for the best on type 2 error.
\newline

The choice for your type 1 error is called the \textbf{significance level}. The corresponding test procedure is called the \textbf{level $\alpha$ test}.

\end{frame}
%----------------------------------------------------------------------------------------






\end{document} 
